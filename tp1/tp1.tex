\documentclass[11pt]{article}

%\renewcommand\familydefault{\sfdefault}
 
%input preamble and macros
\input{preamble}
\input{macros}

 
\usepackage[margin=1in]{geometry} 
\usepackage{amsmath,amsthm,amssymb}
\usepackage{hyperref}
\usepackage[linewidth=1pt]{mdframed}
%\newcommand{\N}{\mathbb{N}}
%\newcommand{\Z}{\mathbb{Z}}

% Named environments (no counters) 
\newenvironment{theorem}[2][Theorem]{\begin{trivlist}
\item[\hskip \labelsep {\bfseries #1}\hskip \labelsep {\bfseries #2.}]}{\end{trivlist}}
\newenvironment{lemma}[2][Lemma]{\begin{trivlist}
\item[\hskip \labelsep {\bfseries #1}\hskip \labelsep {\bfseries #2.}]}{\end{trivlist}}
%\newenvironment{exercise}[2][Exercise]{\begin{trivlist}
%\item[\hskip \labelsep {\bfseries #1}\hskip \labelsep {\bfseries #2.}]}{\end{trivlist}}
\newenvironment{problem}[2][Problem]{\begin{trivlist}
\item[\hskip \labelsep {\bfseries #1}\hskip \labelsep {\bfseries #2.}]}{\end{trivlist}}
\newenvironment{question}[2][Question]{\begin{trivlist}
\item[\hskip \labelsep {\bfseries #1}\hskip \labelsep {\bfseries #2.}]}{\end{trivlist}}
\newenvironment{corollary}[2][Corollary]{\begin{trivlist}
\item[\hskip \labelsep {\bfseries #1}\hskip \labelsep {\bfseries #2.}]}{\end{trivlist}}
 
% Environments with counters
\newtheoremstyle{myplain} {5mm}% ⟨Space above⟩
{3mm}% ⟨Space below⟩
{}% ⟨Body font⟩
{}% ⟨Indent amount⟩
{\bfseries}% ⟨Theorem head font⟩
{.}% ⟨Punctuation after theorem head⟩
{.5em}% ⟨Space after theorem head⟩2
{}% ⟨Theorem head spec (can be left empty, meaning ‘normal’)⟩

\date{}

\theoremstyle{myplain}
\newtheorem{exercise}{Exercise}

\theoremstyle{definition} % no italics
\newtheorem{subexercise}{}[exercise]

\newcommand{\subex}[1]{\begin{subexercise}#1\end{subexercise}}

\begin{document}
 
% --------------------------------------------------------------
%                         Start here
% --------------------------------------------------------------
 
\title{Modelling and analysis of a cyber-physical system \\\small{Practical Assignment 1}}
\author{Renato Neves}
 
\maketitle

\section*{First Part}
The first goal of the assignment is to \emph{model} and
\emph{analyse} a system that ensures the \emph{correct} functioning of
traffic lights at a T-junction. The latter connects a ``major'' and a
``minor'' road and is depicted below (together with the respective
traffic lights):
\begin{center}
\includegraphics[scale=0.3]{images/tjunction.png}
\end{center}
In this scenario vehicles drive on the left side of the road
and the cross in the picture represents a sensor that tells
whether a car is waiting in the minor road or not.

In order to guarantee a reasonable traffic flow, the system has 
the following constraints:
\begin{enumerate}
\item The lights on the major road will be always set on green, and red
	on the minor road \emph{unless} a vehicle is detected by the sensor.
\item In the latter case, the lights will switch in the standard
  manner and allow traffic to leave the minor road.  After a suitable
  time interval (30s), the lights will revert to their default
  position so that traffic can flow on the major road again.
\item Finally, as soon as a vehicle is detected by the sensor the
  latter is disabled until the minor-road lights are on red again.
\end{enumerate}

The system also respects the following \emph{temporal} constraints:
\begin{enumerate}
\item Interim lights stay on for 5s.
\item There exists 1s delay between switching one light off and the
  other on.
\item The major-road light must stay on green for at least 30s in each
  polling cycle, but must respond to the sensor immediately after
  that.
\end{enumerate}
\underline{The first part of the students' assignment}:
\begin{enumerate}
\item Model in \textsc{UPPAAL} the system of traffic lights described
  previously;
\item Express in \textsc{CTL} the following \emph{reachability}
  properties and test them in \textsc{UPPAAL}: (1) the minor-road
  light can go green; (2) the major-road light can go red.
\item Express in \textsc{CTL} the following \emph{safety} properties
  and test them in \textsc{UPPAAL}: (3) the system never enters in a
  deadlock state; (4) the minor-road and major-road lights cannot be
  green at the same time.
\item Express in \textsc{CTL} the following \emph{liveness} property
  and test it in \textsc{UPPAAL}: (5) if there are cars waiting they
  will \emph{eventually} have green light.
\item Can you think of other desirable properties?  If so please register
  them and check whether they hold or not.
\end{enumerate}

\section*{Second Part}

The previous system of traffic lights works reasonably well under the
assumption that one of the roads has more traffic than the other. But
such an assumption is often \emph{too strong}: it may be the case that
both roads have the same amount of traffic, or even that their traffic
flow varies drastically throughout the day. The second part of this
assignment (more exploratory) aims to address precisely this problem
which is well-known to have significant impact in the economy and the
environment~\footnote{\url{https://ourworld.unu.edu/en/green-idea-self-organizing-traffic-signals}}. To
this effect, we can now assume that \underline{each traffic light has
  a smart sensor attached to it}. The sensor informs whether the
traffic near the light is \textsf{high}, \textsf{low}, or simply
\textsf{no}n-existent.

\medskip
\noindent
\underline{The second part of the assignment}:
\begin{enumerate}
\item Adapt your previous UPPAAL model to take into account the
  information provided by the sensors. One expects, for example, that
  if the rightmost sensor outputs \textsf{high} and the other sensors
  output \textsf{no} then the rightmost traffic light should be on
  green at least until the sensors provide new information.
\item Verify that all the properties mentioned in the first part
  of the assignment still hold.
\item (Valorisation) Note that the second part of the assigment is of
  a more exploratory nature, and thus we give freedom to adjust sensor
  parameters as seen fit in order to promote different and creative
  solutions. We will value properties expressed in CTL that say
  something about the efficiency of the system developed by the
  students.  Such a property can be for example, ``If the rightmost
  sensor always detects \textsf{high} traffic and the others detect
  \textsf{no} traffic at all, then we will observe at most one change
  in the traffic lights''.
\item Write a \underline{report for the first and second part of the
    assignment} that explains (1) your design choices, (2) your models,
  (3) the formulae that you used for benchmarking your systems, and (4)
  the conclusions obtained.
\end{enumerate}

\section*{Submission instructions}

\begin{mdframed}
  \myparagraph{What to submit:} The report in PDF \emph{and} the
  respective 
  UPPAAL models. Send by email (\underline{nevrenato@gmail.com})
  a unique zip file ``\bash{cpp2324-N1_N2.zip}'', where \bash{N1} and
  \bash{N2} are your student numbers. The subject of the email should be
  ``\bash{cpp2324 N1 N2}''.

\myparagraph{Deadline:} 07 Apr 2024 @ 23h59
\end{mdframed}


\end{document}
