\documentclass[a4paper, 11pt]{article}

%% packages
\usepackage{fullpage} % changes the margin
\usepackage{hyperref} % Links
\usepackage[utf8]{inputenc}
\usepackage{lmodern}
\usepackage{amsfonts}
\usepackage{amsthm}
\usepackage{amsmath}
\usepackage{braket}
%%

%% macros
\newcommand{\complex}{\mathbb{C}}
\newcommand{\vecs}{\mathcal{V}}
\newcommand{\id}{\mathrm{id}}
%% environments
\theoremstyle{definition}
\newtheorem{definition}{Definition}
\newtheorem{examples}{Example}
\newtheorem{exercises}{Exercises}
\newtheorem{exercise}{Exercise}
\newtheorem{postulate}{Postulate}
\newtheoremstyle{sub}% name
{2pt}        % Space above, empty = `usual value'
{2pt}        % Space below
{}              % Body font
{}    % Indent amount (empty = no indent, \parindent = para indent)
{}              % Thm head font
{.}    % Punctuation after thm head
{0.5em}         % Space after thm head: \newline = linebreak
{{\thmname{#1}\thmnumber{ #2}}\thmnote{ (#3)}}  % Thm head spec
\theoremstyle{sub}
\newtheorem{subexercise}{Part}[exercise]
\usepackage[linewidth=1pt]{mdframed}

%% config
\date{}
\linespread{1.10}
%%

\begin{document}

\title{Cyber-Physical Programming \\ \large TPC-1}
\author{Renato Neves \\ \scriptsize
  \href{mailto:nevrenato@di.uminho.pt}{nevrenato@di.uminho.pt}}
\maketitle

\begin{exercise}
  Consider the CCS processes $c . (a. \mathtt{0} \parallel b. \mathtt{0})$ and
  $\mathtt{rec}\ X.\ (a.X + a.b.X)$. 
  \begin{subexercise}
    Informally describe what they do.
  \end{subexercise}
  \begin{subexercise}
    Present their transition systems using the semantic rules provided in the
    lectures.
  \end{subexercise} 
\end{exercise}

\begin{exercise}
  Consider the following scenario. There exist  four processes $P_1,\dots,P_4$,
  each of them responsible for performing a certain task repetitively. For
  example $P_1$ might read the current velocity, $P_2$ the current altitude,
  $P_3$ current radiation levels, etc\dots These processes (re)start their
  tasks in increasing order ($P_1$ then $P_2$ etc \dots) but can finish in any
  order. Note as well that process $P_1$ can restart its task only when all
  processes $P_1,\dots,P_4$ finish their current tasks.  Let us thus consider
  the process $P = (I \parallel S \parallel P_1 \parallel \dots \parallel P_4)
  \backslash \{st_1,\dots,st_4,end\}$ where,
  \begin{eqnarray*}
    & I  & = \overline{st_1} \dots \overline{st_4} . \mathtt{0} \\
    & S & = \mathtt{rec}\ X.\ end.end.end.end. \overline{st_1} \dots \overline{st_4}. X \\
    & P_i & = \mathtt{rec}\ Y_i .\ st_i . a_i . b_i . \overline{end} . Y_i \qquad \qquad \qquad
            (1 \leq i \leq 4)
  \end{eqnarray*}
  \begin{subexercise}
    Explain why process $P$ corresponds (or not) to the description above.
  \end{subexercise}
  \begin{subexercise}
    Process $S$ acts a \emph{central scheduler} that coordinates the processes
    $P_1,\dots,P_4$. Rewrite $P$ so that it does not rely on a central
    scheduler and explain the reasoning behind your refactoring.
  \end{subexercise}
  \begin{subexercise}[***]
    Use the tool \href{https://www.mcrl2.org/web/user_manual/index.html}{mCRL2}
    to further explore this scenario, \emph{formally} discussing properties
    that the system already has as well as limitations and possible
    improvements.
  \end{subexercise}
\end{exercise}

\begin{mdframed}
  What to submit: A report in \texttt{PDF} with the solutions of the exercises.
  Please send it by email (\texttt{nevrenato@di.uminho.pt}) with the name
  ``\texttt{cpp-N.pdf}'', where ``\texttt{N}'' is your student number.  The
  subject of the email should be ``\texttt{cpp-N-TPC-1}''.
\end{mdframed}



%% Bibliography
\bibliographystyle{alpha}
\bibliography{biblioTeaching}

\end{document}
